
% Default to the notebook output style

    


% Inherit from the specified cell style.




    
\documentclass[11pt]{article}

    
    
    \usepackage[T1]{fontenc}
    % Nicer default font (+ math font) than Computer Modern for most use cases
    \usepackage{mathpazo}

    % Basic figure setup, for now with no caption control since it's done
    % automatically by Pandoc (which extracts ![](path) syntax from Markdown).
    \usepackage{graphicx}
    % We will generate all images so they have a width \maxwidth. This means
    % that they will get their normal width if they fit onto the page, but
    % are scaled down if they would overflow the margins.
    \makeatletter
    \def\maxwidth{\ifdim\Gin@nat@width>\linewidth\linewidth
    \else\Gin@nat@width\fi}
    \makeatother
    \let\Oldincludegraphics\includegraphics
    % Set max figure width to be 80% of text width, for now hardcoded.
    \renewcommand{\includegraphics}[1]{\Oldincludegraphics[width=.8\maxwidth]{#1}}
    % Ensure that by default, figures have no caption (until we provide a
    % proper Figure object with a Caption API and a way to capture that
    % in the conversion process - todo).
    \usepackage{caption}
    \DeclareCaptionLabelFormat{nolabel}{}
    \captionsetup{labelformat=nolabel}

    \usepackage{adjustbox} % Used to constrain images to a maximum size 
    \usepackage{xcolor} % Allow colors to be defined
    \usepackage{enumerate} % Needed for markdown enumerations to work
    \usepackage{geometry} % Used to adjust the document margins
    \usepackage{amsmath} % Equations
    \usepackage{amssymb} % Equations
    \usepackage{textcomp} % defines textquotesingle
    % Hack from http://tex.stackexchange.com/a/47451/13684:
    \AtBeginDocument{%
        \def\PYZsq{\textquotesingle}% Upright quotes in Pygmentized code
    }
    \usepackage{upquote} % Upright quotes for verbatim code
    \usepackage{eurosym} % defines \euro
    \usepackage[mathletters]{ucs} % Extended unicode (utf-8) support
    \usepackage[utf8x]{inputenc} % Allow utf-8 characters in the tex document
    \usepackage{fancyvrb} % verbatim replacement that allows latex
    \usepackage{grffile} % extends the file name processing of package graphics 
                         % to support a larger range 
    % The hyperref package gives us a pdf with properly built
    % internal navigation ('pdf bookmarks' for the table of contents,
    % internal cross-reference links, web links for URLs, etc.)
    \usepackage{hyperref}
    \usepackage{longtable} % longtable support required by pandoc >1.10
    \usepackage{booktabs}  % table support for pandoc > 1.12.2
    \usepackage[inline]{enumitem} % IRkernel/repr support (it uses the enumerate* environment)
    \usepackage[normalem]{ulem} % ulem is needed to support strikethroughs (\sout)
                                % normalem makes italics be italics, not underlines
    

    
    
    % Colors for the hyperref package
    \definecolor{urlcolor}{rgb}{0,.145,.698}
    \definecolor{linkcolor}{rgb}{.71,0.21,0.01}
    \definecolor{citecolor}{rgb}{.12,.54,.11}

    % ANSI colors
    \definecolor{ansi-black}{HTML}{3E424D}
    \definecolor{ansi-black-intense}{HTML}{282C36}
    \definecolor{ansi-red}{HTML}{E75C58}
    \definecolor{ansi-red-intense}{HTML}{B22B31}
    \definecolor{ansi-green}{HTML}{00A250}
    \definecolor{ansi-green-intense}{HTML}{007427}
    \definecolor{ansi-yellow}{HTML}{DDB62B}
    \definecolor{ansi-yellow-intense}{HTML}{B27D12}
    \definecolor{ansi-blue}{HTML}{208FFB}
    \definecolor{ansi-blue-intense}{HTML}{0065CA}
    \definecolor{ansi-magenta}{HTML}{D160C4}
    \definecolor{ansi-magenta-intense}{HTML}{A03196}
    \definecolor{ansi-cyan}{HTML}{60C6C8}
    \definecolor{ansi-cyan-intense}{HTML}{258F8F}
    \definecolor{ansi-white}{HTML}{C5C1B4}
    \definecolor{ansi-white-intense}{HTML}{A1A6B2}

    % commands and environments needed by pandoc snippets
    % extracted from the output of `pandoc -s`
    \providecommand{\tightlist}{%
      \setlength{\itemsep}{0pt}\setlength{\parskip}{0pt}}
    \DefineVerbatimEnvironment{Highlighting}{Verbatim}{commandchars=\\\{\}}
    % Add ',fontsize=\small' for more characters per line
    \newenvironment{Shaded}{}{}
    \newcommand{\KeywordTok}[1]{\textcolor[rgb]{0.00,0.44,0.13}{\textbf{{#1}}}}
    \newcommand{\DataTypeTok}[1]{\textcolor[rgb]{0.56,0.13,0.00}{{#1}}}
    \newcommand{\DecValTok}[1]{\textcolor[rgb]{0.25,0.63,0.44}{{#1}}}
    \newcommand{\BaseNTok}[1]{\textcolor[rgb]{0.25,0.63,0.44}{{#1}}}
    \newcommand{\FloatTok}[1]{\textcolor[rgb]{0.25,0.63,0.44}{{#1}}}
    \newcommand{\CharTok}[1]{\textcolor[rgb]{0.25,0.44,0.63}{{#1}}}
    \newcommand{\StringTok}[1]{\textcolor[rgb]{0.25,0.44,0.63}{{#1}}}
    \newcommand{\CommentTok}[1]{\textcolor[rgb]{0.38,0.63,0.69}{\textit{{#1}}}}
    \newcommand{\OtherTok}[1]{\textcolor[rgb]{0.00,0.44,0.13}{{#1}}}
    \newcommand{\AlertTok}[1]{\textcolor[rgb]{1.00,0.00,0.00}{\textbf{{#1}}}}
    \newcommand{\FunctionTok}[1]{\textcolor[rgb]{0.02,0.16,0.49}{{#1}}}
    \newcommand{\RegionMarkerTok}[1]{{#1}}
    \newcommand{\ErrorTok}[1]{\textcolor[rgb]{1.00,0.00,0.00}{\textbf{{#1}}}}
    \newcommand{\NormalTok}[1]{{#1}}
    
    % Additional commands for more recent versions of Pandoc
    \newcommand{\ConstantTok}[1]{\textcolor[rgb]{0.53,0.00,0.00}{{#1}}}
    \newcommand{\SpecialCharTok}[1]{\textcolor[rgb]{0.25,0.44,0.63}{{#1}}}
    \newcommand{\VerbatimStringTok}[1]{\textcolor[rgb]{0.25,0.44,0.63}{{#1}}}
    \newcommand{\SpecialStringTok}[1]{\textcolor[rgb]{0.73,0.40,0.53}{{#1}}}
    \newcommand{\ImportTok}[1]{{#1}}
    \newcommand{\DocumentationTok}[1]{\textcolor[rgb]{0.73,0.13,0.13}{\textit{{#1}}}}
    \newcommand{\AnnotationTok}[1]{\textcolor[rgb]{0.38,0.63,0.69}{\textbf{\textit{{#1}}}}}
    \newcommand{\CommentVarTok}[1]{\textcolor[rgb]{0.38,0.63,0.69}{\textbf{\textit{{#1}}}}}
    \newcommand{\VariableTok}[1]{\textcolor[rgb]{0.10,0.09,0.49}{{#1}}}
    \newcommand{\ControlFlowTok}[1]{\textcolor[rgb]{0.00,0.44,0.13}{\textbf{{#1}}}}
    \newcommand{\OperatorTok}[1]{\textcolor[rgb]{0.40,0.40,0.40}{{#1}}}
    \newcommand{\BuiltInTok}[1]{{#1}}
    \newcommand{\ExtensionTok}[1]{{#1}}
    \newcommand{\PreprocessorTok}[1]{\textcolor[rgb]{0.74,0.48,0.00}{{#1}}}
    \newcommand{\AttributeTok}[1]{\textcolor[rgb]{0.49,0.56,0.16}{{#1}}}
    \newcommand{\InformationTok}[1]{\textcolor[rgb]{0.38,0.63,0.69}{\textbf{\textit{{#1}}}}}
    \newcommand{\WarningTok}[1]{\textcolor[rgb]{0.38,0.63,0.69}{\textbf{\textit{{#1}}}}}
    
    
    % Define a nice break command that doesn't care if a line doesn't already
    % exist.
    \def\br{\hspace*{\fill} \\* }
    % Math Jax compatability definitions
    \def\gt{>}
    \def\lt{<}
    % Document parameters
    \title{Bakery Project Report}
    
    
    

    % Pygments definitions
    
\makeatletter
\def\PY@reset{\let\PY@it=\relax \let\PY@bf=\relax%
    \let\PY@ul=\relax \let\PY@tc=\relax%
    \let\PY@bc=\relax \let\PY@ff=\relax}
\def\PY@tok#1{\csname PY@tok@#1\endcsname}
\def\PY@toks#1+{\ifx\relax#1\empty\else%
    \PY@tok{#1}\expandafter\PY@toks\fi}
\def\PY@do#1{\PY@bc{\PY@tc{\PY@ul{%
    \PY@it{\PY@bf{\PY@ff{#1}}}}}}}
\def\PY#1#2{\PY@reset\PY@toks#1+\relax+\PY@do{#2}}

\expandafter\def\csname PY@tok@w\endcsname{\def\PY@tc##1{\textcolor[rgb]{0.73,0.73,0.73}{##1}}}
\expandafter\def\csname PY@tok@c\endcsname{\let\PY@it=\textit\def\PY@tc##1{\textcolor[rgb]{0.25,0.50,0.50}{##1}}}
\expandafter\def\csname PY@tok@cp\endcsname{\def\PY@tc##1{\textcolor[rgb]{0.74,0.48,0.00}{##1}}}
\expandafter\def\csname PY@tok@k\endcsname{\let\PY@bf=\textbf\def\PY@tc##1{\textcolor[rgb]{0.00,0.50,0.00}{##1}}}
\expandafter\def\csname PY@tok@kp\endcsname{\def\PY@tc##1{\textcolor[rgb]{0.00,0.50,0.00}{##1}}}
\expandafter\def\csname PY@tok@kt\endcsname{\def\PY@tc##1{\textcolor[rgb]{0.69,0.00,0.25}{##1}}}
\expandafter\def\csname PY@tok@o\endcsname{\def\PY@tc##1{\textcolor[rgb]{0.40,0.40,0.40}{##1}}}
\expandafter\def\csname PY@tok@ow\endcsname{\let\PY@bf=\textbf\def\PY@tc##1{\textcolor[rgb]{0.67,0.13,1.00}{##1}}}
\expandafter\def\csname PY@tok@nb\endcsname{\def\PY@tc##1{\textcolor[rgb]{0.00,0.50,0.00}{##1}}}
\expandafter\def\csname PY@tok@nf\endcsname{\def\PY@tc##1{\textcolor[rgb]{0.00,0.00,1.00}{##1}}}
\expandafter\def\csname PY@tok@nc\endcsname{\let\PY@bf=\textbf\def\PY@tc##1{\textcolor[rgb]{0.00,0.00,1.00}{##1}}}
\expandafter\def\csname PY@tok@nn\endcsname{\let\PY@bf=\textbf\def\PY@tc##1{\textcolor[rgb]{0.00,0.00,1.00}{##1}}}
\expandafter\def\csname PY@tok@ne\endcsname{\let\PY@bf=\textbf\def\PY@tc##1{\textcolor[rgb]{0.82,0.25,0.23}{##1}}}
\expandafter\def\csname PY@tok@nv\endcsname{\def\PY@tc##1{\textcolor[rgb]{0.10,0.09,0.49}{##1}}}
\expandafter\def\csname PY@tok@no\endcsname{\def\PY@tc##1{\textcolor[rgb]{0.53,0.00,0.00}{##1}}}
\expandafter\def\csname PY@tok@nl\endcsname{\def\PY@tc##1{\textcolor[rgb]{0.63,0.63,0.00}{##1}}}
\expandafter\def\csname PY@tok@ni\endcsname{\let\PY@bf=\textbf\def\PY@tc##1{\textcolor[rgb]{0.60,0.60,0.60}{##1}}}
\expandafter\def\csname PY@tok@na\endcsname{\def\PY@tc##1{\textcolor[rgb]{0.49,0.56,0.16}{##1}}}
\expandafter\def\csname PY@tok@nt\endcsname{\let\PY@bf=\textbf\def\PY@tc##1{\textcolor[rgb]{0.00,0.50,0.00}{##1}}}
\expandafter\def\csname PY@tok@nd\endcsname{\def\PY@tc##1{\textcolor[rgb]{0.67,0.13,1.00}{##1}}}
\expandafter\def\csname PY@tok@s\endcsname{\def\PY@tc##1{\textcolor[rgb]{0.73,0.13,0.13}{##1}}}
\expandafter\def\csname PY@tok@sd\endcsname{\let\PY@it=\textit\def\PY@tc##1{\textcolor[rgb]{0.73,0.13,0.13}{##1}}}
\expandafter\def\csname PY@tok@si\endcsname{\let\PY@bf=\textbf\def\PY@tc##1{\textcolor[rgb]{0.73,0.40,0.53}{##1}}}
\expandafter\def\csname PY@tok@se\endcsname{\let\PY@bf=\textbf\def\PY@tc##1{\textcolor[rgb]{0.73,0.40,0.13}{##1}}}
\expandafter\def\csname PY@tok@sr\endcsname{\def\PY@tc##1{\textcolor[rgb]{0.73,0.40,0.53}{##1}}}
\expandafter\def\csname PY@tok@ss\endcsname{\def\PY@tc##1{\textcolor[rgb]{0.10,0.09,0.49}{##1}}}
\expandafter\def\csname PY@tok@sx\endcsname{\def\PY@tc##1{\textcolor[rgb]{0.00,0.50,0.00}{##1}}}
\expandafter\def\csname PY@tok@m\endcsname{\def\PY@tc##1{\textcolor[rgb]{0.40,0.40,0.40}{##1}}}
\expandafter\def\csname PY@tok@gh\endcsname{\let\PY@bf=\textbf\def\PY@tc##1{\textcolor[rgb]{0.00,0.00,0.50}{##1}}}
\expandafter\def\csname PY@tok@gu\endcsname{\let\PY@bf=\textbf\def\PY@tc##1{\textcolor[rgb]{0.50,0.00,0.50}{##1}}}
\expandafter\def\csname PY@tok@gd\endcsname{\def\PY@tc##1{\textcolor[rgb]{0.63,0.00,0.00}{##1}}}
\expandafter\def\csname PY@tok@gi\endcsname{\def\PY@tc##1{\textcolor[rgb]{0.00,0.63,0.00}{##1}}}
\expandafter\def\csname PY@tok@gr\endcsname{\def\PY@tc##1{\textcolor[rgb]{1.00,0.00,0.00}{##1}}}
\expandafter\def\csname PY@tok@ge\endcsname{\let\PY@it=\textit}
\expandafter\def\csname PY@tok@gs\endcsname{\let\PY@bf=\textbf}
\expandafter\def\csname PY@tok@gp\endcsname{\let\PY@bf=\textbf\def\PY@tc##1{\textcolor[rgb]{0.00,0.00,0.50}{##1}}}
\expandafter\def\csname PY@tok@go\endcsname{\def\PY@tc##1{\textcolor[rgb]{0.53,0.53,0.53}{##1}}}
\expandafter\def\csname PY@tok@gt\endcsname{\def\PY@tc##1{\textcolor[rgb]{0.00,0.27,0.87}{##1}}}
\expandafter\def\csname PY@tok@err\endcsname{\def\PY@bc##1{\setlength{\fboxsep}{0pt}\fcolorbox[rgb]{1.00,0.00,0.00}{1,1,1}{\strut ##1}}}
\expandafter\def\csname PY@tok@kc\endcsname{\let\PY@bf=\textbf\def\PY@tc##1{\textcolor[rgb]{0.00,0.50,0.00}{##1}}}
\expandafter\def\csname PY@tok@kd\endcsname{\let\PY@bf=\textbf\def\PY@tc##1{\textcolor[rgb]{0.00,0.50,0.00}{##1}}}
\expandafter\def\csname PY@tok@kn\endcsname{\let\PY@bf=\textbf\def\PY@tc##1{\textcolor[rgb]{0.00,0.50,0.00}{##1}}}
\expandafter\def\csname PY@tok@kr\endcsname{\let\PY@bf=\textbf\def\PY@tc##1{\textcolor[rgb]{0.00,0.50,0.00}{##1}}}
\expandafter\def\csname PY@tok@bp\endcsname{\def\PY@tc##1{\textcolor[rgb]{0.00,0.50,0.00}{##1}}}
\expandafter\def\csname PY@tok@fm\endcsname{\def\PY@tc##1{\textcolor[rgb]{0.00,0.00,1.00}{##1}}}
\expandafter\def\csname PY@tok@vc\endcsname{\def\PY@tc##1{\textcolor[rgb]{0.10,0.09,0.49}{##1}}}
\expandafter\def\csname PY@tok@vg\endcsname{\def\PY@tc##1{\textcolor[rgb]{0.10,0.09,0.49}{##1}}}
\expandafter\def\csname PY@tok@vi\endcsname{\def\PY@tc##1{\textcolor[rgb]{0.10,0.09,0.49}{##1}}}
\expandafter\def\csname PY@tok@vm\endcsname{\def\PY@tc##1{\textcolor[rgb]{0.10,0.09,0.49}{##1}}}
\expandafter\def\csname PY@tok@sa\endcsname{\def\PY@tc##1{\textcolor[rgb]{0.73,0.13,0.13}{##1}}}
\expandafter\def\csname PY@tok@sb\endcsname{\def\PY@tc##1{\textcolor[rgb]{0.73,0.13,0.13}{##1}}}
\expandafter\def\csname PY@tok@sc\endcsname{\def\PY@tc##1{\textcolor[rgb]{0.73,0.13,0.13}{##1}}}
\expandafter\def\csname PY@tok@dl\endcsname{\def\PY@tc##1{\textcolor[rgb]{0.73,0.13,0.13}{##1}}}
\expandafter\def\csname PY@tok@s2\endcsname{\def\PY@tc##1{\textcolor[rgb]{0.73,0.13,0.13}{##1}}}
\expandafter\def\csname PY@tok@sh\endcsname{\def\PY@tc##1{\textcolor[rgb]{0.73,0.13,0.13}{##1}}}
\expandafter\def\csname PY@tok@s1\endcsname{\def\PY@tc##1{\textcolor[rgb]{0.73,0.13,0.13}{##1}}}
\expandafter\def\csname PY@tok@mb\endcsname{\def\PY@tc##1{\textcolor[rgb]{0.40,0.40,0.40}{##1}}}
\expandafter\def\csname PY@tok@mf\endcsname{\def\PY@tc##1{\textcolor[rgb]{0.40,0.40,0.40}{##1}}}
\expandafter\def\csname PY@tok@mh\endcsname{\def\PY@tc##1{\textcolor[rgb]{0.40,0.40,0.40}{##1}}}
\expandafter\def\csname PY@tok@mi\endcsname{\def\PY@tc##1{\textcolor[rgb]{0.40,0.40,0.40}{##1}}}
\expandafter\def\csname PY@tok@il\endcsname{\def\PY@tc##1{\textcolor[rgb]{0.40,0.40,0.40}{##1}}}
\expandafter\def\csname PY@tok@mo\endcsname{\def\PY@tc##1{\textcolor[rgb]{0.40,0.40,0.40}{##1}}}
\expandafter\def\csname PY@tok@ch\endcsname{\let\PY@it=\textit\def\PY@tc##1{\textcolor[rgb]{0.25,0.50,0.50}{##1}}}
\expandafter\def\csname PY@tok@cm\endcsname{\let\PY@it=\textit\def\PY@tc##1{\textcolor[rgb]{0.25,0.50,0.50}{##1}}}
\expandafter\def\csname PY@tok@cpf\endcsname{\let\PY@it=\textit\def\PY@tc##1{\textcolor[rgb]{0.25,0.50,0.50}{##1}}}
\expandafter\def\csname PY@tok@c1\endcsname{\let\PY@it=\textit\def\PY@tc##1{\textcolor[rgb]{0.25,0.50,0.50}{##1}}}
\expandafter\def\csname PY@tok@cs\endcsname{\let\PY@it=\textit\def\PY@tc##1{\textcolor[rgb]{0.25,0.50,0.50}{##1}}}

\def\PYZbs{\char`\\}
\def\PYZus{\char`\_}
\def\PYZob{\char`\{}
\def\PYZcb{\char`\}}
\def\PYZca{\char`\^}
\def\PYZam{\char`\&}
\def\PYZlt{\char`\<}
\def\PYZgt{\char`\>}
\def\PYZsh{\char`\#}
\def\PYZpc{\char`\%}
\def\PYZdl{\char`\$}
\def\PYZhy{\char`\-}
\def\PYZsq{\char`\'}
\def\PYZdq{\char`\"}
\def\PYZti{\char`\~}
% for compatibility with earlier versions
\def\PYZat{@}
\def\PYZlb{[}
\def\PYZrb{]}
\makeatother


    % Exact colors from NB
    \definecolor{incolor}{rgb}{0.0, 0.0, 0.5}
    \definecolor{outcolor}{rgb}{0.545, 0.0, 0.0}



    
    % Prevent overflowing lines due to hard-to-break entities
    \sloppy 
    % Setup hyperref package
    \hypersetup{
      breaklinks=true,  % so long urls are correctly broken across lines
      colorlinks=true,
      urlcolor=urlcolor,
      linkcolor=linkcolor,
      citecolor=citecolor,
      }
    % Slightly bigger margins than the latex defaults
    
    \geometry{verbose,tmargin=1in,bmargin=1in,lmargin=1in,rmargin=1in}
    
    

    \begin{document}
    
    
    \maketitle
    
    

    
    \begin{verbatim}
Authors:    Mingjie Ye, yemj1017@gmail.com, (+34)658745860
            Luqi Guan, estela96guan@gmail.com, (+34)655061067
\end{verbatim}

\hypertarget{introduction}{%
\section{Introduction}\label{introduction}}

Every businessmen including this bakery desire higher profits. There are
two ways to gain profits: increasing revenues and decreasing expenses.
Considering the relatively low-cost products, it is better to increase
the sales by setting competitive prices in order to increase revenues.
Given the dataset of transactions of a bakery from 30/10/2016 to
09/04/2017, we set our business goals, do analysis and data mining,
therefore propose some advisable staregies according to the results.
Finally, we estimate the increaced profits and loss based on hypothesis
because do not have enough information about the bakery except
transactions.

    \hypertarget{business-goal}{%
\section{Business goal}\label{business-goal}}

The utimate goal of the bakery is to increase profits, namely to
increase revenues and to reduce expenses. In order to raise revenues, we
decide to boost sales rather than increasing prices because the products
such as bread and coffee are price-sensitive. Details about our
strategies will be illustrated in the analysis part below.

    \hypertarget{data-analysis-and-business-strategies}{%
\section{Data analysis and business
strategies}\label{data-analysis-and-business-strategies}}

We choose python as our analysis tool because of the following reasons:
1. This is not a big data problem, the scale of data is acceptable with
python. So we don't need big data tools in this project. 2. The
significant factor of choosing Python is the variety of data science
libraries such as pandas, numpy, matplotlib and seaborn. 3. The
widespread and involved community promotes easy access for us to find
different solutions of similar problems. 4. Data visualization in python
is beautiful and easy to operate. 5. We are familiar with python.

    \hypertarget{inspect-our-dataset}{%
\subsection{Inspect our dataset}\label{inspect-our-dataset}}

Import libraries and files as Bak. See the structure and head.

    \begin{Verbatim}[commandchars=\\\{\}]
{\color{incolor}In [{\color{incolor}12}]:} \PY{k+kn}{import} \PY{n+nn}{numpy} \PY{k}{as} \PY{n+nn}{np} \PY{c+c1}{\PYZsh{} linear algebra}
         \PY{k+kn}{import} \PY{n+nn}{pandas} \PY{k}{as} \PY{n+nn}{pd} \PY{c+c1}{\PYZsh{} data processing, CSV file I/O (e.g. pd.read\PYZus{}csv)}
         \PY{k+kn}{import} \PY{n+nn}{seaborn} \PY{k}{as} \PY{n+nn}{sns}
         \PY{k+kn}{import} \PY{n+nn}{matplotlib}\PY{n+nn}{.}\PY{n+nn}{pyplot} \PY{k}{as} \PY{n+nn}{plt}
         \PY{k+kn}{import} \PY{n+nn}{os}
         \PY{c+c1}{\PYZsh{} import the data}
         \PY{n}{Bak}\PY{o}{=}\PY{n}{pd}\PY{o}{.}\PY{n}{read\PYZus{}csv}\PY{p}{(}\PY{l+s+s1}{\PYZsq{}}\PY{l+s+s1}{./input/BreadBasket\PYZus{}DMS.csv}\PY{l+s+s1}{\PYZsq{}}\PY{p}{)}
         \PY{n}{Bak}\PY{o}{.}\PY{n}{head}\PY{p}{(}\PY{p}{)}
\end{Verbatim}


\begin{Verbatim}[commandchars=\\\{\}]
{\color{outcolor}Out[{\color{outcolor}12}]:}          Date      Time  Transaction           Item
         0  2016-10-30  09:58:11            1          Bread
         1  2016-10-30  10:05:34            2   Scandinavian
         2  2016-10-30  10:05:34            2   Scandinavian
         3  2016-10-30  10:07:57            3  Hot chocolate
         4  2016-10-30  10:07:57            3            Jam
\end{Verbatim}
            
    Drop NONE values and create new columns representing Year, Month, Day
and Hour by Column ``Date''. New head is showed below.

    \begin{Verbatim}[commandchars=\\\{\}]
{\color{incolor}In [{\color{incolor}17}]:} \PY{c+c1}{\PYZsh{} Inspect the data}
         \PY{n}{Bak}\PY{o}{.}\PY{n}{loc}\PY{p}{[}\PY{n}{Bak}\PY{p}{[}\PY{l+s+s1}{\PYZsq{}}\PY{l+s+s1}{Item}\PY{l+s+s1}{\PYZsq{}}\PY{p}{]}\PY{o}{==}\PY{l+s+s1}{\PYZsq{}}\PY{l+s+s1}{NONE}\PY{l+s+s1}{\PYZsq{}}\PY{p}{,}\PY{p}{:}\PY{p}{]}\PY{o}{.}\PY{n}{head}\PY{p}{(}\PY{p}{)}
         \PY{n}{Bak}\PY{o}{.}\PY{n}{loc}\PY{p}{[}\PY{n}{Bak}\PY{p}{[}\PY{l+s+s1}{\PYZsq{}}\PY{l+s+s1}{Item}\PY{l+s+s1}{\PYZsq{}}\PY{p}{]}\PY{o}{==}\PY{l+s+s1}{\PYZsq{}}\PY{l+s+s1}{NONE}\PY{l+s+s1}{\PYZsq{}}\PY{p}{,}\PY{p}{:}\PY{p}{]}\PY{o}{.}\PY{n}{count}\PY{p}{(}\PY{p}{)}
         \PY{c+c1}{\PYZsh{} Drop none values from the dataset}
         \PY{n}{Bak}\PY{o}{=}\PY{n}{Bak}\PY{o}{.}\PY{n}{drop}\PY{p}{(}\PY{n}{Bak}\PY{o}{.}\PY{n}{loc}\PY{p}{[}\PY{n}{Bak}\PY{p}{[}\PY{l+s+s1}{\PYZsq{}}\PY{l+s+s1}{Item}\PY{l+s+s1}{\PYZsq{}}\PY{p}{]}\PY{o}{==}\PY{l+s+s1}{\PYZsq{}}\PY{l+s+s1}{NONE}\PY{l+s+s1}{\PYZsq{}}\PY{p}{]}\PY{o}{.}\PY{n}{index}\PY{p}{)}
         \PY{n}{Bak}\PY{p}{[}\PY{l+s+s1}{\PYZsq{}}\PY{l+s+s1}{Year}\PY{l+s+s1}{\PYZsq{}}\PY{p}{]} \PY{o}{=} \PY{n}{Bak}\PY{o}{.}\PY{n}{Date}\PY{o}{.}\PY{n}{apply}\PY{p}{(}\PY{k}{lambda} \PY{n}{x}\PY{p}{:}\PY{n}{x}\PY{o}{.}\PY{n}{split}\PY{p}{(}\PY{l+s+s1}{\PYZsq{}}\PY{l+s+s1}{\PYZhy{}}\PY{l+s+s1}{\PYZsq{}}\PY{p}{)}\PY{p}{[}\PY{l+m+mi}{0}\PY{p}{]}\PY{p}{)}
         \PY{n}{Bak}\PY{p}{[}\PY{l+s+s1}{\PYZsq{}}\PY{l+s+s1}{Month}\PY{l+s+s1}{\PYZsq{}}\PY{p}{]} \PY{o}{=} \PY{n}{Bak}\PY{o}{.}\PY{n}{Date}\PY{o}{.}\PY{n}{apply}\PY{p}{(}\PY{k}{lambda} \PY{n}{x}\PY{p}{:}\PY{n}{x}\PY{o}{.}\PY{n}{split}\PY{p}{(}\PY{l+s+s1}{\PYZsq{}}\PY{l+s+s1}{\PYZhy{}}\PY{l+s+s1}{\PYZsq{}}\PY{p}{)}\PY{p}{[}\PY{l+m+mi}{1}\PY{p}{]}\PY{p}{)}
         \PY{n}{Bak}\PY{p}{[}\PY{l+s+s1}{\PYZsq{}}\PY{l+s+s1}{Day}\PY{l+s+s1}{\PYZsq{}}\PY{p}{]} \PY{o}{=} \PY{n}{Bak}\PY{o}{.}\PY{n}{Date}\PY{o}{.}\PY{n}{apply}\PY{p}{(}\PY{k}{lambda} \PY{n}{x}\PY{p}{:}\PY{n}{x}\PY{o}{.}\PY{n}{split}\PY{p}{(}\PY{l+s+s1}{\PYZsq{}}\PY{l+s+s1}{\PYZhy{}}\PY{l+s+s1}{\PYZsq{}}\PY{p}{)}\PY{p}{[}\PY{l+m+mi}{2}\PY{p}{]}\PY{p}{)}
         \PY{n}{Bak}\PY{p}{[}\PY{l+s+s1}{\PYZsq{}}\PY{l+s+s1}{Hour}\PY{l+s+s1}{\PYZsq{}}\PY{p}{]} \PY{o}{=}\PY{n}{Bak}\PY{o}{.}\PY{n}{Time}\PY{o}{.}\PY{n}{apply}\PY{p}{(}\PY{k}{lambda} \PY{n}{x}\PY{p}{:}\PY{n+nb}{int}\PY{p}{(}\PY{n}{x}\PY{o}{.}\PY{n}{split}\PY{p}{(}\PY{l+s+s1}{\PYZsq{}}\PY{l+s+s1}{:}\PY{l+s+s1}{\PYZsq{}}\PY{p}{)}\PY{p}{[}\PY{l+m+mi}{0}\PY{p}{]}\PY{p}{)}\PY{p}{)}
         \PY{c+c1}{\PYZsh{}df = df.drop(columns=\PYZsq{}Time\PYZsq{})}
         \PY{n}{Bak}\PY{o}{.}\PY{n}{head}\PY{p}{(}\PY{p}{)}
\end{Verbatim}


\begin{Verbatim}[commandchars=\\\{\}]
{\color{outcolor}Out[{\color{outcolor}17}]:}          Date      Time  Transaction           Item  Year Month Day  Hour
         0  2016-10-30  09:58:11            1          Bread  2016    10  30     9
         1  2016-10-30  10:05:34            2   Scandinavian  2016    10  30    10
         2  2016-10-30  10:05:34            2   Scandinavian  2016    10  30    10
         3  2016-10-30  10:07:57            3  Hot chocolate  2016    10  30    10
         4  2016-10-30  10:07:57            3            Jam  2016    10  30    10
\end{Verbatim}
            
    \begin{Verbatim}[commandchars=\\\{\}]
{\color{incolor}In [{\color{incolor}6}]:} \PY{n+nb}{print}\PY{p}{(}\PY{l+s+s1}{\PYZsq{}}\PY{l+s+s1}{Total number of Items sold at the bakery is:}\PY{l+s+s1}{\PYZsq{}}\PY{p}{,}\PY{n}{Bak}\PY{p}{[}\PY{l+s+s1}{\PYZsq{}}\PY{l+s+s1}{Item}\PY{l+s+s1}{\PYZsq{}}\PY{p}{]}\PY{o}{.}\PY{n}{nunique}\PY{p}{(}\PY{p}{)}\PY{p}{)}
        \PY{n+nb}{print}\PY{p}{(}\PY{l+s+s1}{\PYZsq{}}\PY{l+s+s1}{Ten Most Sold Items At The Bakery}\PY{l+s+s1}{\PYZsq{}}\PY{p}{)}
        \PY{n+nb}{print}\PY{p}{(}\PY{n}{Bak}\PY{p}{[}\PY{l+s+s1}{\PYZsq{}}\PY{l+s+s1}{Item}\PY{l+s+s1}{\PYZsq{}}\PY{p}{]}\PY{o}{.}\PY{n}{value\PYZus{}counts}\PY{p}{(}\PY{p}{)}\PY{o}{.}\PY{n}{head}\PY{p}{(}\PY{l+m+mi}{10}\PY{p}{)}\PY{p}{)}
\end{Verbatim}


    \begin{Verbatim}[commandchars=\\\{\}]
Total number of Items sold at the bakery is: 94
Ten Most Sold Items At The Bakery
Coffee           5471
Bread            3325
Tea              1435
Cake             1025
Pastry            856
Sandwich          771
Medialuna         616
Hot chocolate     590
Cookies           540
Brownie           379
Name: Item, dtype: int64

    \end{Verbatim}

    Show the barplot of 20 most sold items at the Bakery. We could find that
coffee is the most popular item followed by bread, tea and cake.

    \begin{Verbatim}[commandchars=\\\{\}]
{\color{incolor}In [{\color{incolor}18}]:} \PY{n}{fig}\PY{p}{,} \PY{n}{ax}\PY{o}{=}\PY{n}{plt}\PY{o}{.}\PY{n}{subplots}\PY{p}{(}\PY{n}{figsize}\PY{o}{=}\PY{p}{(}\PY{l+m+mi}{16}\PY{p}{,}\PY{l+m+mi}{7}\PY{p}{)}\PY{p}{)}
         \PY{n}{Bak}\PY{p}{[}\PY{l+s+s1}{\PYZsq{}}\PY{l+s+s1}{Item}\PY{l+s+s1}{\PYZsq{}}\PY{p}{]}\PY{o}{.}\PY{n}{value\PYZus{}counts}\PY{p}{(}\PY{p}{)}\PY{o}{.}\PY{n}{sort\PYZus{}values}\PY{p}{(}\PY{n}{ascending}\PY{o}{=}\PY{k+kc}{False}\PY{p}{)}\PY{o}{.}\PY{n}{head}\PY{p}{(}\PY{l+m+mi}{20}\PY{p}{)}\PY{o}{.}\PY{n}{plot}\PY{o}{.}\PY{n}{bar}\PY{p}{(}\PY{n}{width}\PY{o}{=}\PY{l+m+mf}{0.5}\PY{p}{,}\PY{n}{edgecolor}\PY{o}{=}\PY{l+s+s1}{\PYZsq{}}\PY{l+s+s1}{k}\PY{l+s+s1}{\PYZsq{}}\PY{p}{,}\PY{n}{align}\PY{o}{=}\PY{l+s+s1}{\PYZsq{}}\PY{l+s+s1}{center}\PY{l+s+s1}{\PYZsq{}}\PY{p}{,}\PY{n}{linewidth}\PY{o}{=}\PY{l+m+mi}{2}\PY{p}{)}
         \PY{n}{plt}\PY{o}{.}\PY{n}{xlabel}\PY{p}{(}\PY{l+s+s1}{\PYZsq{}}\PY{l+s+s1}{Item}\PY{l+s+s1}{\PYZsq{}}\PY{p}{,}\PY{n}{fontsize}\PY{o}{=}\PY{l+m+mi}{20}\PY{p}{)}
         \PY{n}{plt}\PY{o}{.}\PY{n}{ylabel}\PY{p}{(}\PY{l+s+s1}{\PYZsq{}}\PY{l+s+s1}{Number of transactions}\PY{l+s+s1}{\PYZsq{}}\PY{p}{,}\PY{n}{fontsize}\PY{o}{=}\PY{l+m+mi}{20}\PY{p}{)}
         \PY{n}{ax}\PY{o}{.}\PY{n}{tick\PYZus{}params}\PY{p}{(}\PY{n}{labelsize}\PY{o}{=}\PY{l+m+mi}{20}\PY{p}{)}
         \PY{n}{plt}\PY{o}{.}\PY{n}{title}\PY{p}{(}\PY{l+s+s1}{\PYZsq{}}\PY{l+s+s1}{20 Most Sold Items at the Bakery}\PY{l+s+s1}{\PYZsq{}}\PY{p}{,}\PY{n}{fontsize}\PY{o}{=}\PY{l+m+mi}{25}\PY{p}{)}
         \PY{n}{plt}\PY{o}{.}\PY{n}{grid}\PY{p}{(}\PY{p}{)}
         \PY{n}{plt}\PY{o}{.}\PY{n}{ioff}\PY{p}{(}\PY{p}{)}
         \PY{n}{plt}\PY{o}{.}\PY{n}{show}\PY{p}{(}\PY{p}{)}
\end{Verbatim}


    \begin{center}
    \adjustimage{max size={0.9\linewidth}{0.9\paperheight}}{output_9_0.png}
    \end{center}
    { \hspace*{\fill} \\}
    
    \hypertarget{transactions-per-hour}{%
\subsection{Transactions per hour}\label{transactions-per-hour}}

As the barplot of transactions per hour, we could find that the
transactions generally happen between 8:00 and 18:00, especially around
11 O'clock. It is easy to be explained because people tend to buy the
items at the bakery in the morning and at noon.

Therefore, we could propose our first strategy: \textbf{Open hours of
the bakery could be reduced to 8:00 - 18:00 so that expenses will
decrease. Less employee are required after 16:00.} On the other hand, if
the boss of the bakery don't want close the store so early and is
willing to expand business, \textbf{Offering more types of products
which are popular after 18:00 such as salad, beef, noodles, hamburgers
is another reasonable option.}

    \begin{Verbatim}[commandchars=\\\{\}]
{\color{incolor}In [{\color{incolor}92}]:} \PY{c+c1}{\PYZsh{} use sns to plot the counts bar}
         \PY{n}{plt}\PY{o}{.}\PY{n}{subplots}\PY{p}{(}\PY{n}{figsize}\PY{o}{=}\PY{p}{(}\PY{l+m+mi}{16}\PY{p}{,}\PY{l+m+mi}{7}\PY{p}{)}\PY{p}{)}
         \PY{n}{plt}\PY{o}{.}\PY{n}{title}\PY{p}{(}\PY{l+s+s1}{\PYZsq{}}\PY{l+s+s1}{Transactions per hour}\PY{l+s+s1}{\PYZsq{}}\PY{p}{,}\PY{n}{fontsize}\PY{o}{=}\PY{l+m+mi}{25}\PY{p}{)}
         \PY{n}{trans\PYZus{}hour} \PY{o}{=} \PY{n}{sns}\PY{o}{.}\PY{n}{countplot}\PY{p}{(}\PY{n}{x}\PY{o}{=}\PY{l+s+s2}{\PYZdq{}}\PY{l+s+s2}{Hour}\PY{l+s+s2}{\PYZdq{}}\PY{p}{,} \PY{n}{data}\PY{o}{=}\PY{n}{Bak}\PY{p}{)}
         \PY{n}{plt}\PY{o}{.}\PY{n}{show}\PY{p}{(}\PY{p}{)}
\end{Verbatim}


    \begin{center}
    \adjustimage{max size={0.9\linewidth}{0.9\paperheight}}{output_11_0.png}
    \end{center}
    { \hspace*{\fill} \\}
    
    \hypertarget{transactions-per-month}{%
\subsection{Transactions per month}\label{transactions-per-month}}

    \begin{Verbatim}[commandchars=\\\{\}]
{\color{incolor}In [{\color{incolor}40}]:} \PY{n+nb}{print}\PY{p}{(}\PY{n}{Bak}\PY{o}{.}\PY{n}{groupby}\PY{p}{(}\PY{l+s+s1}{\PYZsq{}}\PY{l+s+s1}{Month}\PY{l+s+s1}{\PYZsq{}}\PY{p}{)}\PY{o}{.}\PY{n}{nunique}\PY{p}{(}\PY{p}{)}\PY{p}{)}
\end{Verbatim}


    \begin{Verbatim}[commandchars=\\\{\}]
       Date  Time  Transaction  Item  Year  Month  Day  Hour
Month                                                       
01       30  1531         1575    49     1      1   30    14
02       28  1591         1630    50     1      1   28    15
03       31  1713         1764    52     1      1   31    11
04        9   502          509    49     1      1    9    14
10        2   175          175    30     1      1    2    10
11       30  2069         2140    58     1      1   30    14
12       29  1633         1672    45     1      1   29    14

    \end{Verbatim}

    By checking the information of every months, we know the counts of
transactions and Days. Then we draw the barplot of transactions per
month. Because of the date of this dataset is from 30/10/2016 to
09/04/2017, April and October are incomplete and considered to be
outliers. There are only 9 days' records available in April and 2 in
October.

    \begin{Verbatim}[commandchars=\\\{\}]
{\color{incolor}In [{\color{incolor}93}]:} \PY{n}{plt}\PY{o}{.}\PY{n}{subplots}\PY{p}{(}\PY{n}{figsize}\PY{o}{=}\PY{p}{(}\PY{l+m+mi}{16}\PY{p}{,}\PY{l+m+mi}{6}\PY{p}{)}\PY{p}{)}
         \PY{n}{plt}\PY{o}{.}\PY{n}{xlabel}\PY{p}{(}\PY{l+s+s1}{\PYZsq{}}\PY{l+s+s1}{Months}\PY{l+s+s1}{\PYZsq{}}\PY{p}{,}\PY{n}{fontsize}\PY{o}{=}\PY{l+m+mi}{18}\PY{p}{)}
         \PY{n}{plt}\PY{o}{.}\PY{n}{ylabel}\PY{p}{(}\PY{l+s+s1}{\PYZsq{}}\PY{l+s+s1}{Number of transactions per month}\PY{l+s+s1}{\PYZsq{}}\PY{p}{,}\PY{n}{fontsize}\PY{o}{=}\PY{l+m+mi}{18}\PY{p}{)}
         \PY{n}{plt}\PY{o}{.}\PY{n}{title}\PY{p}{(}\PY{l+s+s1}{\PYZsq{}}\PY{l+s+s1}{Transactions per month}\PY{l+s+s1}{\PYZsq{}}\PY{p}{,}\PY{n}{fontsize}\PY{o}{=}\PY{l+m+mi}{25}\PY{p}{)}
         \PY{n}{Month\PYZus{}counts} \PY{o}{=} \PY{n}{sns}\PY{o}{.}\PY{n}{countplot}\PY{p}{(}\PY{n}{x}\PY{o}{=}\PY{l+s+s2}{\PYZdq{}}\PY{l+s+s2}{Month}\PY{l+s+s2}{\PYZdq{}}\PY{p}{,} \PY{n}{data}\PY{o}{=}\PY{n}{Bak}\PY{p}{)}
         \PY{n}{plt}\PY{o}{.}\PY{n}{show}\PY{p}{(}\PY{p}{)}
\end{Verbatim}


    \begin{center}
    \adjustimage{max size={0.9\linewidth}{0.9\paperheight}}{output_15_0.png}
    \end{center}
    { \hspace*{\fill} \\}
    
    Considering the number of days in different months, we draw a new plot
of transactions per day in different months by dividing the total counts
by number of days. As the figure showed below, we could see that the
numbers of transactions per day in these months are very close except
October and November. The number in October is the largest but there are
only two days's records and the confidence we think is low. So October
is still a outlier in this figure. However, sales per day in November is
obviously larger than others, which means people are more willing to buy
something in bakery in November. Therefore, we think \textbf{more
activities or discounts could come up in this month to attract more
customers}.

    \begin{Verbatim}[commandchars=\\\{\}]
{\color{incolor}In [{\color{incolor}112}]:} \PY{c+c1}{\PYZsh{} caculate de transactions per day in every months}
          \PY{n}{list\PYZus{}trans} \PY{o}{=} \PY{n}{Bak}\PY{o}{.}\PY{n}{groupby}\PY{p}{(}\PY{l+s+s1}{\PYZsq{}}\PY{l+s+s1}{Month}\PY{l+s+s1}{\PYZsq{}}\PY{p}{)}\PY{p}{[}\PY{l+s+s1}{\PYZsq{}}\PY{l+s+s1}{Transaction}\PY{l+s+s1}{\PYZsq{}}\PY{p}{]}\PY{o}{.}\PY{n}{nunique}\PY{p}{(}\PY{p}{)}\PY{o}{.}\PY{n}{tolist}\PY{p}{(}\PY{p}{)}
          \PY{n}{list\PYZus{}days} \PY{o}{=} \PY{n}{Bak}\PY{o}{.}\PY{n}{groupby}\PY{p}{(}\PY{l+s+s1}{\PYZsq{}}\PY{l+s+s1}{Month}\PY{l+s+s1}{\PYZsq{}}\PY{p}{)}\PY{p}{[}\PY{l+s+s1}{\PYZsq{}}\PY{l+s+s1}{Day}\PY{l+s+s1}{\PYZsq{}}\PY{p}{]}\PY{o}{.}\PY{n}{nunique}\PY{p}{(}\PY{p}{)}\PY{o}{.}\PY{n}{tolist}\PY{p}{(}\PY{p}{)}
          \PY{n}{list\PYZus{}trans\PYZus{}per\PYZus{}day} \PY{o}{=} \PY{p}{[}\PY{n}{x} \PY{o}{/} \PY{n}{y} \PY{k}{for} \PY{n}{x}\PY{p}{,} \PY{n}{y} \PY{o+ow}{in} \PY{n+nb}{zip}\PY{p}{(}\PY{n}{list\PYZus{}trans}\PY{p}{,} \PY{n}{list\PYZus{}days}\PY{p}{)}\PY{p}{]}
          
          \PY{n+nb}{print}\PY{p}{(}\PY{n}{list\PYZus{}trans\PYZus{}per\PYZus{}day}\PY{p}{)}
          \PY{n}{months} \PY{o}{=} \PY{n}{Bak}\PY{o}{.}\PY{n}{groupby}\PY{p}{(}\PY{l+s+s1}{\PYZsq{}}\PY{l+s+s1}{Month}\PY{l+s+s1}{\PYZsq{}}\PY{p}{)}\PY{o}{.}\PY{n}{nunique}\PY{p}{(}\PY{p}{)}\PY{o}{.}\PY{n}{index}\PY{o}{.}\PY{n}{tolist}\PY{p}{(}\PY{p}{)}
          \PY{n}{fig}\PY{p}{,} \PY{n}{Month}\PY{o}{=}\PY{n}{plt}\PY{o}{.}\PY{n}{subplots}\PY{p}{(}\PY{n}{figsize}\PY{o}{=}\PY{p}{(}\PY{l+m+mi}{16}\PY{p}{,}\PY{l+m+mi}{7}\PY{p}{)}\PY{p}{)}
          \PY{n}{Month} \PY{o}{=} \PY{n}{plt}\PY{o}{.}\PY{n}{bar}\PY{p}{(}\PY{n}{months}\PY{p}{,}\PY{n}{list\PYZus{}trans\PYZus{}per\PYZus{}day}\PY{p}{,} \PY{n}{color}\PY{o}{=}\PY{p}{[}\PY{l+s+s1}{\PYZsq{}}\PY{l+s+s1}{c}\PY{l+s+s1}{\PYZsq{}}\PY{p}{]}\PY{p}{,} \PY{n}{edgecolor}\PY{o}{=}\PY{l+s+s1}{\PYZsq{}}\PY{l+s+s1}{k}\PY{l+s+s1}{\PYZsq{}}\PY{p}{)}
          \PY{n}{plt}\PY{o}{.}\PY{n}{xlabel}\PY{p}{(}\PY{l+s+s1}{\PYZsq{}}\PY{l+s+s1}{Months}\PY{l+s+s1}{\PYZsq{}}\PY{p}{,}\PY{n}{fontsize}\PY{o}{=}\PY{l+m+mi}{16}\PY{p}{)}
          \PY{n}{plt}\PY{o}{.}\PY{n}{ylabel}\PY{p}{(}\PY{l+s+s1}{\PYZsq{}}\PY{l+s+s1}{Number of transactions per day}\PY{l+s+s1}{\PYZsq{}}\PY{p}{,}\PY{n}{fontsize}\PY{o}{=}\PY{l+m+mi}{16}\PY{p}{)}
          \PY{n}{plt}\PY{o}{.}\PY{n}{title}\PY{p}{(}\PY{l+s+s1}{\PYZsq{}}\PY{l+s+s1}{Number of transactions per day in different months}\PY{l+s+s1}{\PYZsq{}}\PY{p}{,}\PY{n}{fontsize}\PY{o}{=}\PY{l+m+mi}{25}\PY{p}{)}
          \PY{n}{plt}\PY{o}{.}\PY{n}{grid}\PY{p}{(}\PY{p}{)}
          \PY{n}{plt}\PY{o}{.}\PY{n}{show}\PY{p}{(}\PY{p}{)}
\end{Verbatim}


    \begin{Verbatim}[commandchars=\\\{\}]
[52.5, 58.214285714285715, 56.903225806451616, 56.55555555555556, 87.5, 71.33333333333333, 57.6551724137931]

    \end{Verbatim}

    \begin{center}
    \adjustimage{max size={0.9\linewidth}{0.9\paperheight}}{output_17_1.png}
    \end{center}
    { \hspace*{\fill} \\}
    
    \hypertarget{sales-on-different-days-of-the-week}{%
\subsection{Sales on different days of the
week}\label{sales-on-different-days-of-the-week}}

    \begin{Verbatim}[commandchars=\\\{\}]
{\color{incolor}In [{\color{incolor}132}]:} \PY{c+c1}{\PYZsh{} Sales on different days of the week}
          \PY{n}{Bak1} \PY{o}{=} \PY{n}{Bak}\PY{o}{.}\PY{n}{groupby}\PY{p}{(}\PY{p}{[}\PY{l+s+s1}{\PYZsq{}}\PY{l+s+s1}{Date}\PY{l+s+s1}{\PYZsq{}}\PY{p}{]}\PY{p}{)}\PY{o}{.}\PY{n}{size}\PY{p}{(}\PY{p}{)}\PY{o}{.}\PY{n}{reset\PYZus{}index}\PY{p}{(}\PY{n}{name}\PY{o}{=}\PY{l+s+s1}{\PYZsq{}}\PY{l+s+s1}{counts}\PY{l+s+s1}{\PYZsq{}}\PY{p}{)}
          \PY{n}{Bak1}\PY{p}{[}\PY{l+s+s1}{\PYZsq{}}\PY{l+s+s1}{Day}\PY{l+s+s1}{\PYZsq{}}\PY{p}{]} \PY{o}{=} \PY{n}{pd}\PY{o}{.}\PY{n}{to\PYZus{}datetime}\PY{p}{(}\PY{n}{Bak1}\PY{p}{[}\PY{l+s+s1}{\PYZsq{}}\PY{l+s+s1}{Date}\PY{l+s+s1}{\PYZsq{}}\PY{p}{]}\PY{p}{)}\PY{o}{.}\PY{n}{dt}\PY{o}{.}\PY{n}{day\PYZus{}name}\PY{p}{(}\PY{p}{)}
          
          \PY{n}{plt}\PY{o}{.}\PY{n}{figure}\PY{p}{(}\PY{n}{figsize}\PY{o}{=}\PY{p}{(}\PY{l+m+mi}{24}\PY{p}{,}\PY{l+m+mi}{10}\PY{p}{)}\PY{p}{)}
          \PY{n}{plt}\PY{o}{.}\PY{n}{subplot}\PY{p}{(}\PY{l+m+mi}{2}\PY{p}{,}\PY{l+m+mi}{2}\PY{p}{,}\PY{l+m+mi}{1}\PY{p}{)}
          \PY{n}{ax}\PY{o}{=}\PY{n}{sns}\PY{o}{.}\PY{n}{boxplot}\PY{p}{(}\PY{n}{x}\PY{o}{=}\PY{l+s+s1}{\PYZsq{}}\PY{l+s+s1}{Day}\PY{l+s+s1}{\PYZsq{}}\PY{p}{,}\PY{n}{y}\PY{o}{=}\PY{l+s+s1}{\PYZsq{}}\PY{l+s+s1}{counts}\PY{l+s+s1}{\PYZsq{}}\PY{p}{,}\PY{n}{data}\PY{o}{=}\PY{n}{Bak1}\PY{p}{,}\PY{n}{width}\PY{o}{=}\PY{l+m+mf}{0.8}\PY{p}{,}\PY{n}{linewidth}\PY{o}{=}\PY{l+m+mi}{2}\PY{p}{)}
          \PY{n}{plt}\PY{o}{.}\PY{n}{xlabel}\PY{p}{(}\PY{l+s+s1}{\PYZsq{}}\PY{l+s+s1}{Day of the Week}\PY{l+s+s1}{\PYZsq{}}\PY{p}{,}\PY{n}{fontsize}\PY{o}{=}\PY{l+m+mi}{16}\PY{p}{)}
          \PY{n}{plt}\PY{o}{.}\PY{n}{ylabel}\PY{p}{(}\PY{l+s+s1}{\PYZsq{}}\PY{l+s+s1}{Sales}\PY{l+s+s1}{\PYZsq{}}\PY{p}{,}\PY{n}{fontsize}\PY{o}{=}\PY{l+m+mi}{16}\PY{p}{)}
          \PY{n}{plt}\PY{o}{.}\PY{n}{title}\PY{p}{(}\PY{l+s+s1}{\PYZsq{}}\PY{l+s+s1}{Sales on Different Days of Week}\PY{l+s+s1}{\PYZsq{}}\PY{p}{,}\PY{n}{fontsize}\PY{o}{=}\PY{l+m+mi}{20}\PY{p}{)}
          \PY{n}{ax}\PY{o}{.}\PY{n}{tick\PYZus{}params}\PY{p}{(}\PY{n}{labelsize}\PY{o}{=}\PY{l+m+mi}{10}\PY{p}{)}
          \PY{n}{plt}\PY{o}{.}\PY{n}{grid}\PY{p}{(}\PY{p}{)}
          \PY{n}{plt}\PY{o}{.}\PY{n}{ioff}\PY{p}{(}\PY{p}{)}
          \PY{n}{plt}\PY{o}{.}\PY{n}{show}\PY{p}{(}\PY{p}{)}
\end{Verbatim}


    \begin{center}
    \adjustimage{max size={0.9\linewidth}{0.9\paperheight}}{output_19_0.png}
    \end{center}
    { \hspace*{\fill} \\}
    
    The figure shows the difference among days of week. Obviously, sales on
weekend is much higher than weekday. As for weekdays, the items sell the
most on Friday. This is because people have more times and willings to
shop on these days. In order to increase sales, we could draw our
strategies from two perspectives: \textbf{1. In order to increase the
sales on weekday, we could provide more combos for the staff of
surrounding companies; 2. In order to increase the sales on weekend, we
could offer customers promotional activities, special offers or trial
offers.}

    \hypertarget{apriori-algorithm}{%
\subsection{Apriori Algorithm}\label{apriori-algorithm}}

In order to find the association rules of items in the bakery, we use
apriori algorithm with the help of python package mlxtend.

    \begin{Verbatim}[commandchars=\\\{\}]
{\color{incolor}In [{\color{incolor}139}]:} \PY{c+c1}{\PYZsh{} apriori}
          \PY{k+kn}{from} \PY{n+nn}{mlxtend}\PY{n+nn}{.}\PY{n+nn}{frequent\PYZus{}patterns} \PY{k}{import} \PY{n}{apriori}\PY{p}{,} \PY{n}{association\PYZus{}rules}
          \PY{c+c1}{\PYZsh{} transfrom data to make items as columns and each transaction as a row and count same Items bought in one transaction but fill other cloumns of the row with 0 to represent item which are not bought.}
          \PY{n}{hot\PYZus{}encoded\PYZus{}Bak}\PY{o}{=}\PY{n}{Bak}\PY{o}{.}\PY{n}{groupby}\PY{p}{(}\PY{p}{[}\PY{l+s+s1}{\PYZsq{}}\PY{l+s+s1}{Transaction}\PY{l+s+s1}{\PYZsq{}}\PY{p}{,}\PY{l+s+s1}{\PYZsq{}}\PY{l+s+s1}{Item}\PY{l+s+s1}{\PYZsq{}}\PY{p}{]}\PY{p}{)}\PY{p}{[}\PY{l+s+s1}{\PYZsq{}}\PY{l+s+s1}{Item}\PY{l+s+s1}{\PYZsq{}}\PY{p}{]}\PY{o}{.}\PY{n}{count}\PY{p}{(}\PY{p}{)}\PY{o}{.}\PY{n}{unstack}\PY{p}{(}\PY{p}{)}\PY{o}{.}\PY{n}{reset\PYZus{}index}\PY{p}{(}\PY{p}{)}\PY{o}{.}\PY{n}{fillna}\PY{p}{(}\PY{l+m+mi}{0}\PY{p}{)}\PY{o}{.}\PY{n}{set\PYZus{}index}\PY{p}{(}\PY{l+s+s1}{\PYZsq{}}\PY{l+s+s1}{Transaction}\PY{l+s+s1}{\PYZsq{}}\PY{p}{)}
          \PY{n}{hot\PYZus{}encoded\PYZus{}Bak}\PY{o}{.}\PY{n}{head}\PY{p}{(}\PY{p}{)}
          \PY{n}{hot\PYZus{}encoded\PYZus{}Bak} \PY{o}{=} \PY{n}{hot\PYZus{}encoded\PYZus{}Bak}\PY{o}{.}\PY{n}{applymap}\PY{p}{(}\PY{k}{lambda} \PY{n}{x}\PY{p}{:} \PY{l+m+mi}{0} \PY{k}{if} \PY{n}{x}\PY{o}{\PYZlt{}}\PY{o}{=}\PY{l+m+mi}{0} \PY{k}{else} \PY{l+m+mi}{1}\PY{p}{)}
          
          \PY{c+c1}{\PYZsh{} here we choose the min support as 1\PYZpc{}}
          \PY{n}{frequent\PYZus{}itemsets} \PY{o}{=} \PY{n}{apriori}\PY{p}{(}\PY{n}{hot\PYZus{}encoded\PYZus{}Bak}\PY{p}{,} \PY{n}{min\PYZus{}support}\PY{o}{=}\PY{l+m+mf}{0.01}\PY{p}{,} \PY{n}{use\PYZus{}colnames}\PY{o}{=}\PY{k+kc}{True}\PY{p}{)}
          \PY{n}{rules} \PY{o}{=} \PY{n}{association\PYZus{}rules}\PY{p}{(}\PY{n}{frequent\PYZus{}itemsets}\PY{p}{,} \PY{n}{metric}\PY{o}{=}\PY{l+s+s2}{\PYZdq{}}\PY{l+s+s2}{lift}\PY{l+s+s2}{\PYZdq{}}\PY{p}{,} \PY{n}{min\PYZus{}threshold}\PY{o}{=}\PY{l+m+mi}{1}\PY{p}{)}
          \PY{n}{rules}\PY{o}{.}\PY{n}{head}\PY{p}{(}\PY{l+m+mi}{10}\PY{p}{)}
\end{Verbatim}


\begin{Verbatim}[commandchars=\\\{\}]
{\color{outcolor}Out[{\color{outcolor}139}]:}        antecedents      consequents  antecedent support  consequent support  \textbackslash{}
          0      (Alfajores)         (Coffee)            0.036344            0.478394   
          1         (Coffee)      (Alfajores)            0.478394            0.036344   
          2          (Bread)         (Pastry)            0.327205            0.086107   
          3         (Pastry)          (Bread)            0.086107            0.327205   
          4         (Coffee)        (Brownie)            0.478394            0.040042   
          5        (Brownie)         (Coffee)            0.040042            0.478394   
          6         (Coffee)           (Cake)            0.478394            0.103856   
          7           (Cake)         (Coffee)            0.103856            0.478394   
          8  (Hot chocolate)           (Cake)            0.058320            0.103856   
          9           (Cake)  (Hot chocolate)            0.103856            0.058320   
          
              support  confidence      lift  leverage  conviction  
          0  0.019651    0.540698  1.130235  0.002264    1.135648  
          1  0.019651    0.041078  1.130235  0.002264    1.004936  
          2  0.029160    0.089119  1.034977  0.000985    1.003306  
          3  0.029160    0.338650  1.034977  0.000985    1.017305  
          4  0.019651    0.041078  1.025860  0.000495    1.001080  
          5  0.019651    0.490765  1.025860  0.000495    1.024293  
          6  0.054728    0.114399  1.101515  0.005044    1.011905  
          7  0.054728    0.526958  1.101515  0.005044    1.102664  
          8  0.011410    0.195652  1.883874  0.005354    1.114125  
          9  0.011410    0.109868  1.883874  0.005354    1.057910  
\end{Verbatim}
            
    \begin{Verbatim}[commandchars=\\\{\}]
{\color{incolor}In [{\color{incolor}143}]:} \PY{n}{rules}\PY{p}{[} \PY{p}{(}\PY{n}{rules}\PY{p}{[}\PY{l+s+s1}{\PYZsq{}}\PY{l+s+s1}{lift}\PY{l+s+s1}{\PYZsq{}}\PY{p}{]} \PY{o}{\PYZgt{}}\PY{o}{=} \PY{l+m+mf}{1.15}\PY{p}{)} \PY{o}{\PYZam{}} \PY{p}{(}\PY{n}{rules}\PY{p}{[}\PY{l+s+s1}{\PYZsq{}}\PY{l+s+s1}{confidence}\PY{l+s+s1}{\PYZsq{}}\PY{p}{]} \PY{o}{\PYZgt{}}\PY{o}{=} \PY{l+m+mf}{0.5}\PY{p}{)} \PY{p}{]}
\end{Verbatim}


\begin{Verbatim}[commandchars=\\\{\}]
{\color{outcolor}Out[{\color{outcolor}143}]:}          antecedents consequents  antecedent support  consequent support  \textbackslash{}
          18       (Medialuna)    (Coffee)            0.061807            0.478394   
          23          (Pastry)    (Coffee)            0.086107            0.478394   
          29  (Spanish Brunch)    (Coffee)            0.018172            0.478394   
          31           (Toast)    (Coffee)            0.033597            0.478394   
          
               support  confidence      lift  leverage  conviction  
          18  0.035182    0.569231  1.189878  0.005614    1.210871  
          23  0.047544    0.552147  1.154168  0.006351    1.164682  
          29  0.010882    0.598837  1.251766  0.002189    1.300235  
          31  0.023666    0.704403  1.472431  0.007593    1.764582  
\end{Verbatim}
            
    \begin{itemize}
\tightlist
\item
  Support is an indication of how frequently the item set appears in the
  data set. \[supp(X⇒Y)=\frac{|X \cup Y|}{N}\] N means the total number
  of transactions. Obviously, the more support is, the more useful the
  rule is. However, considering so many types of items and huge scale of
  transactions in this bakery, the support in our case is relatively low
  and we need to find an optimal min\_support. If we set the threshold
  too low, we will get a lot of rules and most of them are not so
  valuable. If we set thethreshold too high, only few rules are
  available. In order to find the most several valuable rules, we set
  our threshold as 1\%.
\item
  For a rule X⇒Y, \textbf{confidence} shows the percentage in which Y is
  bought with X. It's an indication of how often the rule has been found
  to be true. It tells us how likely it is that purchasing X results in
  a purchase of Y. \[conf(X⇒Y)=\frac{supp(X \cup Y)}{supp(X)}\] In our
  case, we only accept the rules whose confidences are no less than
  50\%.
\item
  Lift means how likely item Y is purchased when item X is purchased.
  \[lift(X⇒Y)=\frac{supp(X \cup Y)}{supp(X) supp(Y)}\] A Lift of 1 means
  there is no association between products X and Y. Lift of greater than
  1 means products X and Y are more likely to be bought together.
  Finally, Lift of less than 1 refers to the case where two products are
  unlikely to be bought together. Here, we choose 1.15 as threshold.
  Finally, we get 4 important rules. Therefore, we could \textbf{use
  these association rules by providing menu combo with these two items
  such as morning combo with toast and a cup of coffee}.
\end{itemize}

If this combo strategy works well, it could be expanded with more
association rules as the following shows.

    \begin{Verbatim}[commandchars=\\\{\}]
{\color{incolor}In [{\color{incolor}147}]:} \PY{n}{rules}\PY{p}{[} \PY{p}{(}\PY{n}{rules}\PY{p}{[}\PY{l+s+s1}{\PYZsq{}}\PY{l+s+s1}{lift}\PY{l+s+s1}{\PYZsq{}}\PY{p}{]} \PY{o}{\PYZgt{}}\PY{o}{=} \PY{l+m+mf}{1.1}\PY{p}{)} \PY{o}{\PYZam{}} \PY{p}{(}\PY{n}{rules}\PY{p}{[}\PY{l+s+s1}{\PYZsq{}}\PY{l+s+s1}{confidence}\PY{l+s+s1}{\PYZsq{}}\PY{p}{]} \PY{o}{\PYZgt{}}\PY{o}{=} \PY{l+m+mf}{0.5}\PY{p}{)} \PY{p}{]}
\end{Verbatim}


\begin{Verbatim}[commandchars=\\\{\}]
{\color{outcolor}Out[{\color{outcolor}147}]:}          antecedents consequents  antecedent support  consequent support  \textbackslash{}
          0        (Alfajores)    (Coffee)            0.036344            0.478394   
          7             (Cake)    (Coffee)            0.103856            0.478394   
          16           (Juice)    (Coffee)            0.038563            0.478394   
          18       (Medialuna)    (Coffee)            0.061807            0.478394   
          23          (Pastry)    (Coffee)            0.086107            0.478394   
          24        (Sandwich)    (Coffee)            0.071844            0.478394   
          29  (Spanish Brunch)    (Coffee)            0.018172            0.478394   
          31           (Toast)    (Coffee)            0.033597            0.478394   
          
               support  confidence      lift  leverage  conviction  
          0   0.019651    0.540698  1.130235  0.002264    1.135648  
          7   0.054728    0.526958  1.101515  0.005044    1.102664  
          16  0.020602    0.534247  1.116750  0.002154    1.119919  
          18  0.035182    0.569231  1.189878  0.005614    1.210871  
          23  0.047544    0.552147  1.154168  0.006351    1.164682  
          24  0.038246    0.532353  1.112792  0.003877    1.115384  
          29  0.010882    0.598837  1.251766  0.002189    1.300235  
          31  0.023666    0.704403  1.472431  0.007593    1.764582  
\end{Verbatim}
            
    \hypertarget{sales-by-hour-for-top-10-items}{%
\subsection{Sales by hour for top 10
items}\label{sales-by-hour-for-top-10-items}}

We have analysed the total transactions per hour above. Now we will
analyse this problem in different items. According to the \emph{`Figure:
20 Most Sold Items at the Bakery'}, the total sales mainly remain in the
top 10 items, especially the top two items (coffee and bread).

The below figure shows the sales by hour for top 10 items. We could find
that the peaks of these items are different: 1. Some items have two
peaks, while others have only one peak or the second peaks are not
obvious; 2. The two peaks generally have the same moments. The first is
approximately between 9:00 and 12:00; The second is around 13:00 to
16:00. Combined with the practical situation, it is easy to be explained
that these two peaks are exactly the time of brunch and afternoon tea.
Therefore, we define these two times as brunch time and afternoon tea
time. 3. The items having two peaks (defined as Two-peak Item) are:
coffee, brownie, Cookies and hot chocolate. 4. The items having brunch
time peaks (defined as First-peak Item) are: bread, medialuna and
pastry. 5. The items having afternoon tea time peaks (defined as
Second-peak Item) are: cake, sandwich and tea.

The classification of items support our explanation. Bread, medialuna
and pastry sell well at brunch time. Cake, sandwich and tea sell well in
the afternoon. And the sales of coffee, brownie, cookies and hot
chocolate reach the peaks at both of the times. These results conform to
our real business.

Inspired by the results and conclusion, we could propose some
suggestions to increase the sales:

\begin{enumerate}
\def\labelenumi{\arabic{enumi}.}
\tightlist
\item
  \textbf{Rearrange the positions of different items in the right time.
  For example, at brunch time (9:00-12:00) salesman or saleswoman could
  place more First-peak Items and Two-peak Items while less Second-peak
  Items.}
\item
  \textbf{Produce different items in the right time. This is similar
  with the suggestion 1. Because First-peak Items and Two-peak Items
  sell better at brunch time, it is sensible to produce more these items
  and less Second-peak Items at brunch time.}
\item
  \textbf{Give a discount to First-peak Items or place them into combo
  with other items in the afternoon to ensure that all of the items are
  sold this day. Because the sales decrease after the first peak
  according to the figure.}
\end{enumerate}

    \begin{Verbatim}[commandchars=\\\{\}]
{\color{incolor}In [{\color{incolor}150}]:} \PY{n}{Top\PYZus{}items}\PY{o}{=}\PY{n}{Bak}\PY{p}{[}\PY{l+s+s1}{\PYZsq{}}\PY{l+s+s1}{Item}\PY{l+s+s1}{\PYZsq{}}\PY{p}{]}\PY{o}{.}\PY{n}{value\PYZus{}counts}\PY{p}{(}\PY{p}{)}\PY{o}{.}\PY{n}{head}\PY{p}{(}\PY{l+m+mi}{10}\PY{p}{)}\PY{o}{.}\PY{n}{index}\PY{o}{.}\PY{n}{tolist}\PY{p}{(}\PY{p}{)}
          \PY{n}{Hour\PYZus{}by\PYZus{}Item}\PY{o}{=}\PY{n}{Bak}\PY{p}{[}\PY{p}{[}\PY{l+s+s1}{\PYZsq{}}\PY{l+s+s1}{Hour}\PY{l+s+s1}{\PYZsq{}}\PY{p}{,}\PY{l+s+s1}{\PYZsq{}}\PY{l+s+s1}{Item}\PY{l+s+s1}{\PYZsq{}}\PY{p}{,}\PY{l+s+s1}{\PYZsq{}}\PY{l+s+s1}{Transaction}\PY{l+s+s1}{\PYZsq{}}\PY{p}{]}\PY{p}{]}\PY{o}{.}\PY{n}{groupby}\PY{p}{(}\PY{p}{[}\PY{l+s+s1}{\PYZsq{}}\PY{l+s+s1}{Hour}\PY{l+s+s1}{\PYZsq{}}\PY{p}{,}\PY{l+s+s1}{\PYZsq{}}\PY{l+s+s1}{Item}\PY{l+s+s1}{\PYZsq{}}\PY{p}{]}\PY{p}{,}\PY{n}{as\PYZus{}index}\PY{o}{=}\PY{k+kc}{False}\PY{p}{)}\PY{o}{.}\PY{n}{sum}\PY{p}{(}\PY{p}{)}
          \PY{n}{plt}\PY{o}{.}\PY{n}{figure}\PY{p}{(}\PY{n}{figsize}\PY{o}{=}\PY{p}{[}\PY{l+m+mi}{13}\PY{p}{,}\PY{l+m+mi}{5}\PY{p}{]}\PY{p}{)}
          \PY{n}{plt}\PY{o}{.}\PY{n}{ticklabel\PYZus{}format}\PY{p}{(}\PY{n}{style}\PY{o}{=}\PY{l+s+s1}{\PYZsq{}}\PY{l+s+s1}{plain}\PY{l+s+s1}{\PYZsq{}}\PY{p}{,} \PY{n}{axis}\PY{o}{=}\PY{l+s+s1}{\PYZsq{}}\PY{l+s+s1}{y}\PY{l+s+s1}{\PYZsq{}}\PY{p}{)}
          \PY{n}{plt}\PY{o}{.}\PY{n}{title}\PY{p}{(}\PY{l+s+s1}{\PYZsq{}}\PY{l+s+s1}{Sale by Hour for Top 10 Items}\PY{l+s+s1}{\PYZsq{}}\PY{p}{)}
          \PY{n}{sns}\PY{o}{.}\PY{n}{lineplot}\PY{p}{(}\PY{n}{x}\PY{o}{=}\PY{l+s+s1}{\PYZsq{}}\PY{l+s+s1}{Hour}\PY{l+s+s1}{\PYZsq{}}\PY{p}{,}\PY{n}{y}\PY{o}{=}\PY{l+s+s1}{\PYZsq{}}\PY{l+s+s1}{Transaction}\PY{l+s+s1}{\PYZsq{}}\PY{p}{,}\PY{n}{data}\PY{o}{=}\PY{n}{Hour\PYZus{}by\PYZus{}Item}\PY{p}{[}\PY{n}{Hour\PYZus{}by\PYZus{}Item}\PY{p}{[}\PY{l+s+s1}{\PYZsq{}}\PY{l+s+s1}{Item}\PY{l+s+s1}{\PYZsq{}}\PY{p}{]}\PY{o}{.}\PY{n}{isin}\PY{p}{(}\PY{n}{Top\PYZus{}items}\PY{p}{)}\PY{p}{]}\PY{p}{,}\PY{n}{hue}\PY{o}{=}\PY{l+s+s1}{\PYZsq{}}\PY{l+s+s1}{Item}\PY{l+s+s1}{\PYZsq{}}\PY{p}{)}
          \PY{n}{plt}\PY{o}{.}\PY{n}{show}\PY{p}{(}\PY{p}{)}
\end{Verbatim}


    \begin{center}
    \adjustimage{max size={0.9\linewidth}{0.9\paperheight}}{output_27_0.png}
    \end{center}
    { \hspace*{\fill} \\}
    
    \hypertarget{summary}{%
\section{Summary}\label{summary}}

In this project, we set our business goal as `Increase profits'. There
are two ways: increasing revenues and reducing expenses. Increasing
revenues will be the mainly approach. Considering the business pattern
and low-cost items in bakery, the way to increase revenues is mainly by
increasing sales.

Given the dataset, we chose python as our data analysis and data mining
tools, then analysed the transactions per hour, transactions per month
and sales on different days of the week. Plenty of valuable strategies
were proposed with the analysis. In order to check if there is any
association rules, we used apriori algorithm and got some rules followed
one strategy. Finally, we came into different items to see the sales of
them in different hours. The difference among variable items were
obvious, so we classified them into three different categories based on
the peak of sales curve: Two-peak Item, First-peak Item and Second-peak
Item. Meanwhile several detailed suggestions were proposed.

All the \textbf{10 strategies} we proposed in this project are as
follows:

\begin{enumerate}
\def\labelenumi{\arabic{enumi}.}
\tightlist
\item
  \textbf{Open hours of the bakery could be reduced to 8:00 - 18:00 so
  that expenses will decrease.}
\item
  \textbf{Less employee are required after 16:00.}
\item
  \textbf{If the boss of the bakery don't want close the store so early
  and is willing to expand business, Offering more types of products
  which are popular after 18:00 such as salad, beef, noodles, hamburgers
  is another reasonable option.}
\item
  \textbf{More activities or discounts could come up in November to
  attract more customers.}
\item
  \textbf{In order to increase the sales on weekday, we could provide
  more combos for the staff of surrounding companies.}
\item
  \textbf{In order to increase the sales on weekend, we could offer
  customers promotional activities, special offers or trial offers.}
\item
  \textbf{Use the listed association rules by providing menu combo with
  these two items such as morning combo with toast and a cup of coffee.
  If this combo strategy works well, it could be expanded with more
  association rules as the following shows.}
\item
  \textbf{Rearrange the positions of different items in the right time.
  For example, at brunch time (9:00-12:00) salesman or saleswoman could
  place more First-peak Items and Two-peak Items while less Second-peak
  Items.}
\item
  \textbf{Produce different items in the right time. This is similar
  with the suggestion 1. Because First-peak Items and Two-peak Items
  sell better at brunch time, it is sensible to produce more these items
  and less Second-peak Items at brunch time.}
\item
  \textbf{Give a discount to First-peak Items or place them into combo
  with other items in the afternoon to ensure that all of the items are
  sold this day. Because the sales decrease after the first peak
  according to the figure.}
\end{enumerate}


    % Add a bibliography block to the postdoc
    
    
    
    \end{document}
